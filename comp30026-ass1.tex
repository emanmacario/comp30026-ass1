\documentclass[12pt]{article}
\usepackage{graphicx}
\usepackage{xspace}
\usepackage{color, colortbl}
\usepackage{amssymb}
\pagestyle{empty}
\definecolor{Gray}{gray}{0.9}
\textwidth      165mm
\textheight     240mm
\topmargin      -18mm
\oddsidemargin  -2mm
\evensidemargin 2mm
\newcommand{\impl}{\mathbin{\Rightarrow}}
\newcommand{\biim}{\mathbin{\Leftrightarrow}}
\renewcommand{\theenumi}{\alph{enumi}}

\author{Emmanuel Macario - 831659}
\title{COMP30026 Models of Computation Assignment 1}
\date{September, 2018}

\begin{document}

\maketitle

\subsection*{Challenge 1}
\begin{enumerate}
\item 
$\begin{array}{|ccc|c|c|}
   \hline
   P & Q & R & ((\neg P \land Q) \impl \neg R) \impl R & R
\\ \hline
    0 & 0 & 0 & 0 & 0
\\ 0 & 0 & 1 & 1 & 1
\\ 0 & 1 & 0 & 0 & 0
\\ 0 & 1 & 1 & 1 & 1
\\ 1 & 0 & 0 & 0 & 0
\\ 1 & 0 & 1 & 1 & 1
\\ 1 & 1 & 0 & 0 & 0
\\ 1 & 1 & 1 & 1 & 1
\\ \hline
\end{array}$
$\therefore ((\neg P \land Q) \impl \neg R) \impl R \equiv R$
\bigskip

\item 
$\begin{array}{|ccc|c|c|}
   \hline
   P & Q & R & (P \impl (Q \lor R)) \land (P \biim Q) & P \biim Q
\\ \hline
    0 & 0 & 0 & 1 & 1
\\ 0 & 0 & 1 & 1 & 1
\\ 0 & 1 & 0 & 0 & 0
\\ 0 & 1 & 1 & 0 & 0
\\ 1 & 0 & 0 & 0 & 0
\\ 1 & 0 & 1 & 0 & 0
\\ 1 & 1 & 0 & 1 & 1
\\ 1 & 1 & 1 & 1 & 1
\\ \hline
\end{array}$
$\therefore (P \impl (Q \lor R)) \land (P \biim Q) \equiv P \biim Q$
\bigskip

\item
$\begin{array}{|cc|c|c|}
   \hline
   P & Q & P \impl (P \biim Q) & P \impl Q
\\ \hline
    0 & 0 & 1 & 1
\\ 0 & 1 & 1 & 1
\\ 1 & 0 & 0 & 0
\\ 1 & 1 & 1 & 1
\\ \hline
\end{array}$
$\therefore P \impl (P \biim Q) \equiv P \impl Q$
\bigskip

\item
$\begin{array}{|ccc|c|c|}
   \hline
   P & Q & R & P \lor (R \impl Q) & (Q \impl P) \impl (R \impl P)
\\ \hline
    0 & 0 & 0 & 1 & 1
\\ 0 & 0 & 1 & 0 & 0
\\ 0 & 1 & 0 & 1 & 1
\\ 0 & 1 & 1 & 1 & 1
\\ 1 & 0 & 0 & 1 & 1
\\ 1 & 0 & 1 & 1 & 1
\\ 1 & 1 & 0 & 1 & 1
\\ 1 & 1 & 1 & 1 & 1
\\ \hline
\end{array}$
$\therefore P \lor (R \impl Q) \equiv (Q \impl P) \impl (R \impl P)$
\bigskip

\end{enumerate}


\subsection*{Challenge 2}

Firstly, let us define both inhabitant's statements in propositional logic.
Inhabitant $P$'s statements are equivalent to $\neg Q$ and $Q'$, respectively. 
Inhabitant $Q$'s statements are equivalent to $P$ and $\neg P'$, respectively.

\bigskip
\noindent
We are given information that the statements given by the inhabitants $P$ and $Q$ 
respectively must be either true or false, but \emph{cannot} be a mixture of both. 
In other words, if an inhabitant lies, then every statement they say is guaranteed to 
be a false statement.

\bigskip
\noindent
This means that if inhabitant $P$ is telling the truth, then $\neg Q \land Q'$ is true, 
otherwise $Q \land \neg Q'$ must be true. Similarly, if inhabitant $Q$ is telling the truth,
then $P \land \neg P'$ must be true, otherwise $\neg P \land P'$ must be true.


\bigskip
\noindent
Let $A$ mean $P$ \emph{is telling the truth}, and $B$ mean $Q$ is \emph{telling
the truth}. We can surmise when $P$ and $Q$ are lying or telling the truth via truth tables.

\bigskip
\begin{minipage}{0.5\textwidth}
\begin{center}
$\begin{array}{|cc|c|}
   \hline
   P & P' & A
\\ \hline
   0 & 0 & 1
\\ 0 & 1 & 0
\\ 1 & 0 & 0
\\ 1 & 1 & 1
\\ \hline
\end{array}$
\end{center}
\end{minipage}
\begin{minipage}{0.35\textwidth}
\begin{center}
$\begin{array}{|cc|c|}
   \hline
   Q & Q' & B
\\ \hline
   0 & 0 & 1
\\ 0 & 1 & 0
\\ 1 & 0 & 0
\\ 1 & 1 & 1
\\ \hline
\end{array}$
\end{center}
\end{minipage}

\bigskip
\noindent
It can be seen that $P$ is truthful when $P \biim P'$ is true, and lies when $P 
\oplus P'$ is true. The same notions hold for inhabitant $Q$, with respect to 
literals $Q$ and $Q'$.

\bigskip
\noindent
Now, we can extract the maximal amount of information provided by the two 
inhabitants, in the form of propositional formulas (1) and (2).

\bigskip
\begin{equation}
((P \biim P') \land (\neg Q \land Q')) \oplus ((P \oplus P') \land (Q \land \neg Q'))
\end{equation}
\begin{equation}
((Q \biim Q') \land (P \land \neg P')) \oplus ((Q \oplus Q') \land (\neg P \land P'))
\end{equation}

\bigskip
\noindent
Using a truth table, we can determine for each of $P$ and $Q$ whether they 
are a knight or a knave, and whether they are healthy or sick.

\bigskip
\begin{center}
\resizebox{0.95\textwidth}{!}{$\begin{array}{|cccc|c|c|}
   \hline
   P & P' & Q & Q' & ((P \biim P') \land (\neg Q \land Q')) \oplus ((P \oplus P') \land (Q \land \neg Q')) & 
   ((Q \biim Q') \land (P \land \neg P')) \oplus ((Q \oplus Q') \land (\neg P \land P'))
\\ \hline
   0 & 0 & 0 & 0 & 0 & 0 
\\ 0 & 0 & 0 & 1 & 1 & 0
\\ 0 & 0 & 1 & 0 & 0 & 0
\\ 0 & 0 & 1 & 1 & 0 & 0 
\\ 0 & 1 & 0 & 0 & 0 & 0 
\\ 0 & 1 & 0 & 1 & 0 & 1 
\\ \rowcolor{Gray}
   0 & 1 & 1 & 0 & 1 & 1 
\\ 0 & 1 & 1 & 1 & 0 & 0
\\ 1 & 0 & 0 & 0 & 0 & 1 
\\ 1 & 0 & 0 & 1 & 0 & 0 
\\ 1 & 0 & 1 & 0 & 1 & 0 
\\ 1 & 0 & 1 & 1 & 0 & 1 
\\ 1 & 1 & 0 & 0 & 0 & 0 
\\ 1 & 1 & 0 & 1 & 1 & 0 
\\ 1 & 1 & 1 & 0 & 0 & 0 
\\ 1 & 1 & 1 & 1 & 0 & 0
\\ \hline
\end{array}$}
\end{center}

\bigskip
\noindent
We know that both formulas must be true. Therefore,
there is only \emph{one} possible scenario. That is, $P$
is a \emph{healthy knave} and $Q$ is a \emph{sick knight}.

% CHALLENGE 3
\subsection*{Challenge 3}
\begin{enumerate}
  \item
The two formulas are not equivalent. For example, consider the
interpretation $I$ with domain $D$ = \{0,1\} and $P(0)$. Additionally, 
let $Q$ always be \emph{false}.

\bigskip
\noindent
For this interpretation, it can be seen that $I \vDash \forall x
(P(x)) \impl Q$, but $I \nvDash \forall x(P(x) \impl Q)$.
Hence, $\forall x(P(x)) \impl Q \nvDash \forall x(P(x) \impl Q)$.

\bigskip
\noindent
Therefore, $\forall x (P(x)) \impl Q$ is \emph{not} equivalent to
$\forall x(P(x) \impl Q)$.
\end{enumerate}


% CHALLENGE 4
\subsection*{Challenge 4}
\begin{enumerate}
  \item $\forall x \forall y(M(x,y) \impl F(x))$
  \item $\forall x \forall y((U(x) \land M(y,x)) \impl U(y))$
  \item $\forall x((F(x) \land U(x) \land B(x)) \impl D(x))$
  \item $\forall x \forall y((U(x) \land M(x,y) \land D(x)) \impl D(y))$
  \item $\forall x((S(x) \land U(x)) \impl B(x))$
\end{enumerate}

\bigskip
\noindent
The \emph{Horn clauses} generated are as follows:
\begin{enumerate}
  \item $\{\neg M(x,y), F(x)\}$
  \item $\{\neg U(x), \neg M(y,x), U(y)\}$
  \item $\{\neg F(x), \neg U(x), \neg B(x), D(x)\}$
  \item $\{\neg U(x), \neg M(x), \neg D(x), D(y)\}$
  \item $\{\neg S(x), \neg U(x), B(x)\}$
\end{enumerate}

% CHALLENGE 5
\subsection*{Challenge 5}

\begin{enumerate}
  \item (a1) $\forall x \forall y(N(x,y) \impl N(y,x))\\\\$
        (a2) $\forall x \forall y \exists u((M(u,x) \land \neg M(u,y)) \impl N(x,y))\\\\$
        (a3) $\forall x(\neg \exists u(M(u,x)) \impl E(x))\\\\$
        (a4) $\forall x \forall y \forall u((D(x,y) \land M(u,x)) \impl \neg M(u,y))$
\end{enumerate}


% CHALLENGE 6
\subsection*{Challenge 6}
See Grok.


\end{document}